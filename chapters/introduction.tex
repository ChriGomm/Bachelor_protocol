\chapter{Introduction}
The physics of driven atomic ensembles coupled to external baths are subject of intensive research. Their characteristic of being exposed to environmental influence is a property found in almost all realizations of quantum mechanical atomic systems in nature and experiments. The significance of understanding the behavior of such systems is immense, as well as the number of aspects that can be considered is vast. The introduction of various models has enabled a great amount of learning in this context \cite{diehl_quantum_2008,diehl_dynamical_2010,cabot_metastable_2022,mattes_entangled_2023,krishna_measurement-induced_2023,jin_photon_2013,marcuzzi_absorbing_2016}. The atoms can be directly interacting or not interacting. The inclusion of different external baths can model various coupling types and induce miscellaneous dynamics. Dissipation creating environments can induce decay of exited states and can lead the system to long-time stationary states \cite{camalet_non-equilibrium_2011,huangfu_steady_2018}. On the contrary the well known example of Rabi driving \cite{rabi_space_1937} can enforce an oscillatory behavior on the collection of atoms \cite{dudin_observation_2012}.

The interplay of driving and dissipation has been under investigation in numerous contributions to this field. The exchange of energy, excitations and other quantities between atoms and environment can lead to genuine persistent non-equilibrium states of the system \cite{diehl_quantum_2008,diehl_dynamical_2010,cabot_metastable_2022,mattes_entangled_2023,krishna_measurement-induced_2023,jin_photon_2013,marcuzzi_absorbing_2016}. In recent work it was found that driven-dissipative models can exhibit various interesting phases, %. A very incomplete list of such findings can contain for example metastability \cite{cabot_metastable_2022} or 
resulting from the spontaneous breaking of time-translation symmetry \cite{mattes_entangled_2023,krishna_measurement-induced_2023}. In some contexts the rise of such oscillatory dynamics have been analyzed with respect to synchronization effects \cite{cabot_quantum_2019,cabot_metastable_2021,giorgi_transient_2019,weiner_phase_2017}.\\\\%, or have been part of a separation of time scales \cite{labay-mora_quantum_2023}.\\\\
The breaking of time-translation symmetry is usually related to the dynamics for long times. In a time crystal phase this long-time dynamics - or more correctly the measured observables - are not constant in time, but oscillate with a certain frequency. Hence they are not invariant with respect to a finite shift in time. I want to mention that the existence of a time crystal phase is not limited to the steady state, but can also take place in systems with metastable dynamics \cite{else_prethermal_2017,gambetta_discrete_2019}. There a separation of time scales regarding the dynamical generator is observed. As a consequence the maintenance of a metastable oscillating state can occur over a long period of time, before the system finally relaxes to a steady state. \\Time crystals generally can appear in two variants. The occurrence of discrete time-translation symmetry breaking has been shown both in experimental  \cite{choi_observation_2017,zhang_observation_2017} as well as theoretical work \cite{yao_discrete_2017,sacha_time_2018} to appear in certain periodically driven systems. In this context the system adopts oscillations of a fraction of the driving frequency for long times. Thus the dynamics break the discrete time-translation symmetry of the Hamiltonian. Complementary, continuous time crystals have been experimentally found \cite{kesler_emergent_2019} and theoretically described \cite{tucker_shattered_2018,iemini_boundary_2018,owen_quantum_2018} in the presence of dissipation in form of different decay channels. Here the long-time dynamics of the system can oscillate with frequencies selected from a continuous set of values. 

Additionally to the exploration of new physics, the breaking of time-translation symmetry is investigated for the use in a variety of applications%. Possible implementations are as 
, such as the efficient storage of energy \cite{paulino_thermodynamics_2025} or to stabilize qubits. % or as a measure to enhance the stability of qubits \cite{barnes_stabilization_2019,qiao_floquet-enhanced_2021}. 
Another example of possible applications is the use in time keeping devices \cite{taheri_all-optical_2022}.\\\\
The other concept which is important in my work is synchronization. It describes the alignment of two or more coupled oscillating systems to each other such that the frequencies of the oscillators adapt to a common one \cite{pikovskij_synchronization_2007}. Synchronization is also taking place when an oscillator takes on the frequency of a weakly coupled external force. The alignment of frequencies is referred to as frequency locking. Synchronization in the context of open quantum systems is a young field of research \cite{galve_quantum_2017}. An example for suggested applications was to retrieve information from a system that is difficult to measure through evaluating an accessible system, which is coupled for synchronization to the one of interest \cite{giorgi_probing_2016}. A different opportunity can be to harvest the synchronization of several coupled oscillators, which can reduce the common phase noise well below the individual values \cite{zhang_synchronization_2015,matheny_phase_2014}. This can be of great advantage to applications for time keeping and sensing.\\\\
This thesis will analyze the influence of collective periodic driving combined with a number of interactions of dissipative nature to a system of $N$ non-interacting two-level atoms. The driving will be induced by a detuned laser and the system is subject to collective spontaneous emission and collective dephasing. %, which is a genuine feature of open many body quantum systems. 
In a mean-field analysis the dynamics of the collective spin, an observable of the collective state of the system, will be examined. As will be demonstrated, the resulting model exhibits a featureless fully mixed steady state. However the introduction of local pumping to the atoms can lead to the emergence of a time crystal phase.\\\\
\chref{ch:theory} introduces the theoretical framework that is used in order to describe the open quantum setup. Starting from a collision model approach I will derive the Lindblad master equation and introduce the mean field equations of motion for the considered observables. \chref{ch:mean_field_analysis} constitutes the main part of my thesis and is divided into two main sections. 

In \secref{sec:zero_detuning} I treat the case of zero detuning analytically before I proceed with a purely numerical investigation of the full model in \secref{sec:detuned_analysis}. Here I will find a rich set of dynamics including limit cycles. The behavior of possible long-time states shows a highly non-trivial dependence on the parameter configurations, including multistability. I will also address the possible explanation of the stationary states as resulted from synchronization effects with respect to the laser. As a final chapter I will finish the thesis with a brief summary of my learnings followed by an out-look to possible subsequent investigations.\newpage
% \printbibliography