\chapter{Conclusion}
The subject of this thesis was an ensemble of non-interacting two-level atoms in the thermodynamical limit, coherently driven by a laser. The interaction of the atoms with the environment accounted for collective spontaneous emission processes, collective dephasing and a local pumping. I showed that in the absence of local pumping the system converges to a featureless fully mixed state for long times. Further it was analytically shown that the system converges to stationary non-equilibrium states, when local pumping is introduced. In the case where the laser is resonant to the atoms of the system, a stationary state is always adopted for long times. Here the system oscillates with its intrinsic frequency which is shared by the laser. The thesis was able to demonstrate that in the presence of both, detuning and local pumping, the rise of limit cycles can be observed.% The analysis in this setup was completely performed numerically. 

One of the most intriguing learnings from this project was that the stationary states, which were present for a wide range of parameter configurations of the full model, can be explained via synchronization. In these cases the system's dynamics result from a frequency locking and its oscillations follow the frequency of the laser driving. 

A time crystal phase was discovered where no stable stationary solutions to the equations of motion of the collective spin exist. Within this phase parameter configurations exist, where more than one limit cycle can be adopted for long times and hence multistability arises. This phenomenon was examined with regard to its behavior when parameters were changed. For the number of present limit cycles as well as their average values a rich dependence on the parameter configuration was observed in form of strong discontinuity and repeating patterns.\\\\
The rise of multistability is linked to a transition in the average spin values of a limit cycle. But a comprehensive understanding of this phenomenon is yet to be found. An explanation of the intriguing courses the number of present limit cycles as well as their average values take is an interesting topic for subsequent work. Also the rise of synchronization has to be analyzed in more detail, and can possibly be described through transforming the mean-field equations into a different basis, where a phase is introduced. Maybe a link between multistability and synchronization can be uncovered. Another open question is how the existence of more than one limit cycle manifests itself in the space of the collective spin. A step in this direction can be the analysis of the basin of attraction of the long-time states.

Apart from these open questions, which have been out of scope of the Bachelor project, it would also be interesting to examine the system away from the thermodynamical limit. An important task is to evaluate whether the phenomena found in this work withstand in a finite system. The permutation invariance of the atoms in the considered model can be harvested to make numerical calculations for the quantum Lindblad equations more feasible. Possible synchronization effects and time crystal phases even in the presence of dephasing can have implications for research and applications \cite{nande_integrating_2025}.

% \printbibliography