\documentclass{article}

%\title{Effects of User Interface in Communication Delays in Interplanetary Analog Missions}
\title{Bachelor Arbeit}
\author{Christian Gommeringer}
% \authorpronound{his}
% \prevdega{Your Previous Degree \#1}
% \prevdegb{Your Previous Degree \#2}
% \gradmonth{August}
% \gradyear{2024}
% \department{Your Department}
% \cchair{Your committee chair}
% \cmembera{Your committee member \#1}
% \cmemberb{Your committee member \#2}
% \cmemberc{Your committee member \#3}
% \cmemberd{Your committee member \#4}
% \graddean{Name of Grad Dean}

% \abstract{Your abstract goes here. You can also use an input command with a filename.}
% \acknowledgements{Your acknowledgements go here. You can also use an input command with a filename.}
% \dedication{Your dedication goes here.}
\usepackage{amsmath}
\usepackage{amsfonts}
% \usepackage{unicode-math}
\usepackage{amssymb}
\usepackage{braket}
\usepackage{mathtools}
\usepackage[a4paper, margin=2cm]{geometry}
\usepackage{bbold}
\usepackage{graphicx}
\usepackage[toc,page]{appendix}
\usepackage{wrapfig}
\usepackage{hyperref}
\usepackage{floatrow}
\usepackage{subcaption}
\usepackage[backend=biber]{biblatex} %Imports biblatex package
\addbibresource{source.bib}
% \setmathfont{Asana Math}
% \makeglossaries
% \loadglsentries{glossary}

% Right align bibliography to get rid of under full warnings.
% There are two options for this. Preference is to keep bibliography justified. 
%\apptocmd{\thebibliography}{\raggedright}{}{}
% \apptocmd{\sloppy}{\hbadness 10000\relax}{}{}
% \overfullrule=5pt

\begin{document}
\newcommand{\half}{\frac{1}{2}}
\newcommand{\Tr}{\operatorname{Tr}}
\newcommand{\Trs}[1]{\Tr_S\left\{#1\right\}}
\newcommand{\Trb}[1]{\Tr_B\left\{#1\right\}}
% \newcommand{\dt}[1]{\frac{\text{d} #1}{\text{d}t}}
\newcommand{\dt}{\frac{\text{d}}{\text{d}t}}
\newcommand{\tk}{\tilde{\kappa}}
\newcommand{\tw}{\tilde{\omega}}
\newcommand{\dkw}{\left(\frac{\text{d}}{\text{d}{y}}\,\frac{\Omega^2}{K}\right)}
\newcommand{\kw}{\frac{\Omega^2}{K}}
\newcommand{\ty}{\tilde{y}}
\newcommand{\diff}{\text{d}}
\maketitle


\section{The model}
In this project I will examine a System of N 2-level atoms or spins, described in a collision model framework\cite{ciccarello_cm,cm_source2}. The system will be under the influence of several interaction types due to coupling to external baths. These effects are captured by the collision model Hamiltonian 
\begin{align*}
    H &= \frac{\delta}{2}\,J_z+ \omega\,J_x+\alpha\,(J_+a + J_-a^\dagger)+ \beta\,J_z\,(b^\dagger+b)+\xi\,\sum_{i=1}^N\,(\sigma_i^-c_i^\dagger+\sigma_i^+c_i)\\
    \text{with}\quad H_S&=\frac{\delta}{2}\,J_z+ \omega\,J_x\\
    V_a&=\alpha\,(J_+a- + J_-a^\dagger)\\
    V_b&=\beta\,J_z\,(b^\dagger+b)\\
    V_c&=\xi\,\sum_{i=1}^N\,(\sigma_i^-c_i^\dagger+\sigma_i^+c_i)\\
    J_k &=\half\,\sum_{j=1}^N\sigma_j^k\quad k\in\{x,y,z\}\\
    J_\pm&=\half\,\sum_{j=1}^N\sigma_j^\pm\\
    \alpha&=\sqrt{\frac{\kappa}{\Delta tN}}\,;\quad \beta=\sqrt{\frac{\gamma}{\Delta t}}\,;\quad \xi=\sqrt{\frac{\Gamma}{4\Delta t}}
\end{align*}
The operators acting on the system are the pauli spin matrices $\sigma^{x,y,z}$. I will go through the terms, where I will label them and explain their derivation. $\delta$ labels the detuning of a laser driving coupled with the parameter $\omega$. The responsible field is treated classically, which is possible if the field is strong enough, so that the behavior of the atoms has no influence on the state of this light field. The term can be derived by starting with the light atom interaction in the dipole approximation.
\begin{align*}
    H = \omega_0\,\ket{e}\bra{e}+ \hat{d}\,\cos(\omega_Lt)\quad,
\end{align*} 
where $\hat{d}$ is the dipole operator multiplied with the external field amplitude
\begin{align*}
    \hat{d}&=\hat{\bf{r}}\,\bf{E_0}\\
    &=2\omega\, \sigma^x
\end{align*}
in the bases of the atomic ground and exited state $\ket{g}$, $\ket{e}$ with $2\omega\vcentcolon=\braket{e|\hat{\bf{r}}\,\bf{E_0}|g}$, what is assumed to be real. The elementary charge shall be contained in $\bf{E_0}$. The diagonal elements vanish because of the parity properties of the states. After transforming the state of the system via the gauge transformation with the unitary $U=\exp(i\omega t)\,\ket{e}\bra{e}+\ket{g}\bra{g}$ the Hamiltonian transforms to 
\begin{align*}
    H\rightarrow \,&UHU^\dagger+i\,(\frac{\diff}{\diff t}\,U)\,U^\dagger\\
    =\,&\omega_0\,\ket{e}\bra{e}-\omega_L\,\ket{e}\bra{e} + 2\omega\,\cos(\omega_L t)\,\{e^{i\omega_L t} \ket{e}\bra{g}+e^{-i\omega_L t}\ket{g}\bra{e}\}\\
    =\,&(\omega_0-\omega_L)\,\ket{e}\bra{e}+\omega\,\sigma_x
\end{align*}
Between the second to third step the counter rotating exponentials have been neglected. By resetting the zero energy point to the middle of the atomic states and defining $\delta=\omega_0-\omega_L$ the Hamiltonian can be written as
\begin{align*}
    H = \frac{\delta}{2}\,\sigma^z+ \omega\,\sigma^x
\end{align*}
The Jaynes-Cummings term can be derived in a similar manner with a quantum field. 
\begin{align*}
    H= \frac{\omega_0}{2}\,\sigma^z+\int_{0}^\infty\diff\omega\,\omega\,a^\dagger_\omega a_\omega+ \hat{\boldsymbol{d}}\,\int_{0}^\infty\diff\omega\,\tilde{\alpha}(\omega)\,\boldsymbol{E_0}\,(a_\omega+a^\dagger_\omega)
\end{align*}
where the dipole approximation has already been performed. By assuming that only a small number of modes actually contribute to the interaction, one can without consequences extend the integral to $-\infty$. I will also assume that within the relevant frequency interval the coupling strength $\tilde{\alpha}$ is constant, making it possible to write the Hamiltonian as
\begin{align*}
    H= \frac{\omega_0}{2}\,\sigma^z+\int_{-\infty}^\infty\diff\omega\,\omega\,a^\dagger_\omega a_\omega+ \alpha\,\int_{-\infty}^\infty\diff\omega\,\sigma^x\,(a_\omega+a^\dagger_\omega)
\end{align*}
By going to the interaction picture with respect to the bosonic bath and neglecting counter rotating terms, we get to the equation
\begin{align*}
    H= \frac{\omega_0}{2}\,\sigma^z+ \alpha\,\int_{-\infty}^\infty\diff\omega\,(e^{-i\omega t}\sigma^+a_\omega+e^{i\omega t}\sigma^-a^\dagger_\omega)
\end{align*}
This resembles the Jaynes-Cummings model in the interaction picture. At this point I want to introduce the collision model approach, which will be used for this project. Starting with the definition
\begin{align*}
    a_t\vcentcolon=\int_{-\infty}^{\infty}\diff\omega\,e^{-i\omega t}a_\omega
\end{align*}
lets the Hamiltonian read
\begin{align*}
    H= \frac{\omega_0}{2}\,\sigma^z+ \alpha\,(\sigma^+a_t+\sigma^-a^\dagger_t)
\end{align*}
The unitary propagation can be decomposed into $N$ consecutive propagations of duration $\Delta t$
\begin{align*}
    U(t,t_0)&=\mathcal{T}\left(  \exp(-i\int_{t_0}^t\diff t'\,H(t'))  \right)=U_{N}\cdots U_1\\
    \text{with}\quad U_n&=\mathcal{T}\left( \exp(-i\int_{t_0+(n-1)\,\Delta t}^{t_0+n\,\Delta t}\diff t'\,H(t'))  \right)
\end{align*}
As can be shown\cite{ciccarello_cm} the propagation can be approximated up to the second order in $\Delta t$ via
\begin{align*}
    U_n \approx 1-i\int_{t_0+(n-1)\,\Delta t}^{t_0+n\,\Delta t}\diff t \,H(t) - \half\,\left(\int_{t_0+(n-1)\,\Delta t}^{t_0+n\,\Delta t}\diff t \,H(t)\right)^2
\end{align*}
After introducing
\begin{align*}
    a_n\vcentcolon&=\frac{1}{\sqrt{\Delta t}}\int_{t_0+(n-1)\,\Delta t}^{t_0+n\,\Delta t}\diff t\,a_t
\end{align*}
and redefining $\alpha\rightarrow\alpha/\sqrt{\Delta t}$ this can be written as
\begin{align*}
    U_n &\approx 1-i\,H_n \,\Delta t-\half\,H_n^2\,\Delta t^2\\
    \text{with}\quad H_n&=\alpha\,(\sigma_+a_n+\sigma_-a^\dagger_n)
\end{align*}
The evolution of the system factorizes into consecutive interactions described by the Hamiltonian $H_n$. Those interactions can be interpreted as $N$ independent bath segments, called in the following ancillas, interacting separately with the system, when the following property holds. At each such interaction the state of the bath is completely uncorrelated  to and not influenced by what has happened during the previous interactions - or at least to a sufficient degree. In this case the main criteria for a description through the collision model are met, which are \cite{ciccarello_cm}:
\begin{itemize}
    \item ancillas do not interact with each other 
    \item ancillas are initially uncorrelated 
    \item each ancilla collides with S only once 
\end{itemize}. 
The assumption of independent consecutive interactions leans on the assumption of a correlation time of the bath, which is much smaller than the one of the system. Note, that $[a_n,a^\dagger_{n'}]=\delta_{n,n'}$ holds, which strengthens the notion of independent modes.\\\\%, and therefore the modes of different $n$ do not depend on each other, which is important for being able to describe the system via a collision model with collisions with ancillas $n$.
in our case the ancillas of the collective Jaynes-Cummings interaction $V_a$ are all prepared in the vacuum state, and thus the interaction describes the spontaneous emission of the atoms.\\\\
The Term $V_b$ resembles a dephasing process, which can be simulated via spin-dependent kick operations\cite{exp_spin_boson}. The last interaction $V_c$ introduces another Jaynes-Cummings term, where each spin is individually coupled to a bosonic bath. Each ancilla will be set in the first exited state. Thus this interaction constitutes a local pumping process.

\section{Derivation of the Lindblad equation}
In order to derive the Lindblad equation from the collision we will continue from the propagation by the single ancilla,
\begin{align*}
    U_n &\approx 1-i\,H_n \,\Delta t-\half\,H_n^2\,\Delta t^2
\end{align*}
and decompose the Hamiltonian in a term that solely effect the System and one that describes the interaction with the baths.
\begin{align*}
    H_n &= H_n^S + V_n\\
    H_n^S&=\frac{\delta}{2}\,J_z+\omega\,J_x\\
    V_n&= \alpha\,(J_+a_n + J_-a_n^\dagger)+ \beta\,J_z\,(b_n^\dagger+b_n)+\xi\,\sum_{i=1}^N\,(\sigma_{i}^-c_{n,i}^\dagger+\sigma_i^+c_{n,i})
\end{align*}
Now the propagation of the density matrix $\chi$ of the composite system ($\rho$) and ancilla ($\eta_n$). As assumed they initially are uncorrelated $\chi_n=\rho_n\otimes\eta_n$, where $\rho_n$ is the density matrix of the system at the beginning of the $n$th $\Delta t$-interval.
\begin{align*}
    \rho_{n+1}\otimes\eta_n((n+1)\,\Delta t)&=U_n\chi_nU_n^\dagger
\end{align*}
Here $\eta_m((n+1)\,\Delta t)$ represents the state of the ancilla at the end of the interaction.
If one no plugs in the approximate form of the propagation $U_n$, one reaches at
\begin{align*}
    \rho_{n+1}\otimes\eta_n((n+1)\,\Delta t)-\rho_n\otimes\eta_n&=-i\,[H_n,\chi_n]\,\Delta t+H_n\,\chi_n\,H_n\,\Delta t^2+\half\,[H_n^2,\chi]_+\,\Delta t^2
\end{align*}
where $[\cdot,\cdot]_+$ is the anticommutation relation. If one now neglects the terms quadratic in $H_n^S$, as is explained in \cite{ciccarello_cm}, the equation of motion becomes
\begin{align*}
    \rho_{n+1}\otimes\eta_n((n+1)\,\Delta t)-\rho_n\otimes\eta_n=\Delta\chi_n\vcentcolon=-i\,[H_n,\chi_n]\,\Delta t+V_n\,\chi_n\,V_n\,\Delta t^2+\half\,[V_n^2,\chi]_+\,\Delta t^2
\end{align*}
Tracing out the baths leads to the Lindblad equation for the density matrix. In the following I will drop the subscript $n$. As the the expectation values of the interaction operators on the bath side vanish in our case, the first term becomes
\begin{align*}
    L_S=-i\,[H_S,\rho]
    \text{for}\quad \frac{\Delta\rho}{\Delta t}=L_S+L_a+L_b+L_c
\end{align*}
I will go through the different contributions. 
\begin{align*}
    L_a &= \alpha^2\,\Trb{V_a\chi V_a-\half\,[V_a^2,\chi]_+}\,\Delta t\\
    &= \frac{\kappa}{N}\,J_-\rho J_+\,\Trb{a\,a^\dagger\eta_a}-\half\,\frac{\kappa}{N}\,[J_+J_-,\rho]_+\,\Trb{[a\,a^\dagger,\eta_a]_+}\\
    &=\frac{\kappa}{N}\,(J_-\rho J_+-\half\,[J_+J_-,\rho]_+)\\
    \text{for}\quad\eta_a&=\ket{\text{vac}}_a\bra{\text{vac}}
\end{align*}
From the start I excluded the vanishing terms $\Trb{(a^\dagger a^\dagger+a\,a+a^\dagger a)\cdot\eta}$. Analogously can be continued for the other parts of $H$.
\begin{align*}
    L_b&= \beta^2\,J_z\rho J_z\, \Trb{b\,b^\dagger\eta_b}\,\Delta t-\beta^2\,\half\,[J_z^2,\rho]_+\,\Trb{[b\,b^\dagger,\eta_b]_+}\,\Delta t\\
    &=\gamma\,J_z\rho J_z-\half\,\gamma\,[J_z^2,\rho]_+\\
    \eta_b&=\ket{\text{vac}}_b\bra{\text{vac}}\\
    L_c&= \xi^2\,\sum_i\sigma_i^-\rho \sigma_i^+\, \Trb{c_i^\dagger c_i\eta^c_i}\,\Delta t-\xi^2\,\half\,\sum_i\,[\sigma_i^-\sigma_i^+,\rho]_+\,\Trb{[c_i^\dagger c_i,\eta^c_i]_+}\,\Delta t\\
    &=\frac{\Gamma}{4}\,\left(\sum_i\sigma_i^-\rho \sigma_i^+-\half\,\sum_i\,[\sigma_i^-\sigma_i^+,\rho]_+\right)\\
    \eta^c_i&=\ket{1}^c_i\bra{1}
\end{align*}
If the interaction with each ancilla gets very short, one can approximately perform the limit $\Delta t\rightarrow0$ and propagate the density matrix of the system continuously through time. The time evolution of the expectation value of an observable $O_S$ can be determined via 
\begin{align*}
    \dt \braket{O_S}=\Trs{O_S\,\dt\rho}&=\Trs{O_S\,(L_S+L_a+L_b+L_c)}\\
    &=\vcentcolon \mathcal{L}_S(O_S)+\mathcal{L}_a(O_S)+\mathcal{L}_b(O_S)+\mathcal{L}_c(O_S)
\end{align*}
With this expression the time evolution of the the spin components can be computed.
\begin{align*}
    \mathcal{L}_S(J_z)&=\omega\,\braket{J_y}\\
    \mathcal{L}_S(J_\pm)&=-i\,\left(\mp\frac{\delta}{2}\, \braket{J_\pm} \pm \omega\, \braket{J_z}\right)
\end{align*}
The interactions with the bath are described in the Lindblad form. The detailed calculations have be shifted to the Appendix \ref{appendix:eqm_derv}. Here I only collect the different contributions to the Lindblad equation for the different interaction types.
\begin{align*}
    \mathcal{L}_a(J_z)=-\frac{\kappa}{N}\,\Trs{J_+ J_- \rho},\quad
    \mathcal{L}_a(J_+)=\frac{\kappa}{N}\,\Trs{J_+ J_z \rho},\quad
    \mathcal{L}_a(J_-)=\frac{\kappa}{N}\,\Trs{J_z J_- \rho}
\end{align*}
\begin{align*}
    \mathcal{L}_b(J_z)=0,\quad
    \mathcal{L}_b(J_\pm)=-\half\,\gamma\,\Trs{J_\pm\rho}
\end{align*}
\begin{align*}
    \mathcal{L}_c(J_z)=\half\,N\,\Gamma-\Gamma\,\braket{J_z},\quad
    \mathcal{L}_c(J_-)=-\frac{\Gamma}{2}\,\sum_{k=1}^N\Trs{\sigma_k^-\rho},\quad
    \mathcal{L}_c(J_+)=-\frac{\Gamma}{2}\,\sum_{k=1}^N\Trs{\sigma_k^+\rho}
\end{align*}
Putting everything together the equations of motion read
\begin{align*}
    \dt \braket{J_+}&=\,i\,\frac{\delta}{2}\,\braket{J_+}-i\,\omega\,\braket{J_z}-\half\,(\gamma+\Gamma)\,\braket{J_+}+\frac{\kappa}{N}\,\braket{J_+ J_z}\\
    \dt \braket{J_-}&=-i\,\frac{\delta}{2}\,\braket{J_-}+i\,\omega\,\braket{J_z}-\half\,(\gamma+\Gamma)\,\braket{J_-}+\frac{\kappa}{N}\,\braket{J_z J_-}\\
    &=-i\,\frac{\delta}{2}\,\braket{J_-}+i\,\omega\,\braket{J_z}-\half\,(\gamma+\Gamma)\,\braket{J_-}+\frac{\kappa}{N}\,\braket{J_- J_z}-\frac{\kappa}{N}\,\braket{J_-}\\
    \dt \braket{J_z}&=\omega\,\braket{J_y} - \frac{\kappa}{N}\,\braket{J_+ J_-}+\half\,N\,\Gamma-\Gamma\,\braket{J_z}
\end{align*}
These can be transformed from the ladder operators back to the spin components.
\begin{align*}
    \dt\,\braket{J_x}=\half\,\left(\dt \braket{J_+}+\dt \braket{J_-}\right)&=-\frac{\delta}{2}\,\braket{J_y}-\half\,(\gamma+\Gamma)\,\braket{J_x}+\frac{\kappa}{N}\,\braket{J_xJ_z}-\frac{\kappa}{2N}\,\braket{J_-}\\
    \dt\,\braket{J_y}=-i\,\half\,\left(\dt \braket{J_+}-\dt \braket{J_-}\right)&=\frac{\delta}{2}\,\braket{J_x}-\omega\,\braket{J_z}-\half\,(\gamma+\Gamma)\,\braket{J_y}+\frac{\kappa}{N}\,\braket{J_yJ_z}-i\,\frac{\kappa}{2N}\,\braket{J_-}\\
    \dt \braket{J_z}&=\omega\,\braket{J_y} - \frac{\kappa}{N}\,\braket{J_x^2+J_y^2} -\frac{\kappa}{N}\,\braket{J_z}+\half\,N\,\Gamma-\Gamma\,\braket{J_z}
\end{align*}
From here I will continue with the mean field approximation $\braket{A\,B}=\braket{A}\,\braket{B}$, and define the mean field expectation values $m_\alpha=\braket{J_\alpha}/N$. I will take the thermodynamic limit $N\rightarrow\infty$, in which the mean field treatment is correct. In this limit terms of the form $\braket{J_\alpha}/N^2\rightarrow0$ vanish. The mean field equations of motion then read
\begin{align*}
    \dt m_x &= -\frac{\delta}{2}\,m_y-\half\,(\gamma+\Gamma)\,m_x+\kappa\,m_x m_z\\
    \dt m_y &= \frac{\delta}{2}\,m_x-\omega\,m_z-\half\,(\gamma+\Gamma)\,m_y+\kappa\,m_y m_z\\
    \dt m_z &= \omega\,m_z - \kappa\,(m_x^2+m_y^2)+\half\Gamma-\Gamma\,m_z
\end{align*}
\section{Evaluating the mean field equations}
As a first step I will examine properties of the total spin
\begin{align*}
    m^2\vcentcolon&=m_x^2+m_y^2+m_z^2\\
    \Rightarrow\quad\dt m^2&=2\,\left( m_x\,\dt m_x +m_y\,\dt m_y +m_z\,\dt m_z \right)\\
    &=-\vec{m}^t\,\left( \begin{array}{ccc}
        \Gamma+\gamma & 0&0  \\
        0& \Gamma+\gamma & 0\\
        0&0&2\,\Gamma
   \end{array}\right)\,\vec{m}+\Gamma\,m_z\\
   &=-\left(  (\Gamma+\gamma)\,(m_x^2+m_y^2)+2\,\Gamma\,m_z^2-\Gamma\,m_z \right)
\end{align*}
From this expression one can already say that laser driving, detuning and spontaneous emission have no direct influence on the total spin modulus. Further can be recognized that in the case, where there is no pumping present, the dephasing term drives the system into a fully mixed state, which follows from the negative definiteness of the derivative of total spin modulus. The proof of this intuitively understandable statement can be found in the Appendix \ref{appendix:msq_calc}. Therefore for interesting things to happen in this model, a pumping process has to be implemented.

\subsection{Zero detuning}
We proceed by solving the set of mean field equations for stationary states, i.e. $\text{d}/\text{d}t\,m_x=\text{d}/\text{d}t\,m_y=\text{d}/\text{d}t\,m_z=0$. The $m_z$-equation can be reformulated as
\begin{align*}
    m_z=\half-\frac{1}{\Gamma}\,\left(\kappa\,(m_x^2+ m_y^2)-\omega\,m_y  \right)
\end{align*}
in the case $\Gamma\neq0$. This makes the problem 2 dimensional.
\begin{align*}
    -\frac{\delta}{2}\,m_y-\half\,(\gamma+\Gamma)\,m_x+\kappa\,m_x\,\left( \half-\frac{1}{\Gamma}\,\left(\kappa\,(m_x^2+ m_y^2)-\omega\,m_y  \right)  \right)&=0\\
    \frac{\delta}{2}\,m_x-\omega\,\left( \half-\frac{1}{\Gamma}\,\left(\kappa\,(m_x^2+ m_y^2)-\omega\,m_y  \right)  \right)&\\-\half\,(\gamma+\Gamma)\,m_y+\kappa\,m_y\,\left( \half-\frac{1}{\Gamma}\,\left(\kappa\,(m_x^2+ m_y^2)-\omega\,m_y  \right)  \right)&=0\\\\
    \text{multiplying with }2\Gamma\text{ yields}\quad\quad\quad\hspace*{6cm}&
    \\\Rightarrow\quad-\Gamma\delta\,m_y-\Gamma\,(\gamma+\Gamma-\kappa)\,m_x-2\,\kappa^2\,m_x\,( m_x^2+ m_y^2)+2\,\kappa\omega\,m_xm_y  &=0\\\\
    \Gamma\delta\,m_x-\omega\Gamma-\Gamma\,(\gamma+\Gamma-\kappa+2\,\omega^2/\Gamma)\,m_y&\\
    -2\,\kappa^2\,m_y\,( m_x^2+ m_y^2)+2\,\kappa\omega\,(m_x^2+2\,m_y^2)  &=0
\end{align*}
In the following I set $\delta=0$. With the definition $\omega=\tilde{\omega}/\sqrt{2}$, $\kappa=\tilde{\kappa}/\sqrt{2}$ and $\Gamma+\gamma-\tilde{\kappa}/\sqrt{2}=\vcentcolon \tilde{A}\tilde{\kappa}^2$, we are looking at the equations
\begin{align*}
    -m_x\,\left(\Gamma\tilde{A}+( m_x^2+ m_y^2)-\frac{\tilde{\omega}}{\tilde{\kappa}}\,m_y\right)  &=0\\\\
    -\frac{\tilde{\omega}\Gamma}{\sqrt{2}\,\tilde{\kappa}^2}-\Gamma\tilde{A}\,m_y    -m_ym_x^2- m_y^3+\frac{\tilde{\omega}}{\tilde{\kappa}}\,(m_x^2+2\,m_y^2)-\frac{\tilde{\omega}^2}{\tilde{\kappa}^2}\,m_y  &=0
\end{align*}
First I show, that there is no solution for $m_x\neq0$. This would require
\begin{align*}
    m_x^2&=-m_y^2+\frac{\tilde{\omega}}{\tilde{\kappa}}\,m_y-\Gamma\tilde{A} \\
    \Rightarrow\quad0&=-\frac{\tilde{\omega}\Gamma}{\sqrt{2}\,\tilde{\kappa}^2}-\Gamma\tilde{A}\,m_y    -m_y\,(-m_y^2+\frac{\tilde{\omega}}{\tilde{\kappa}}\,m_y-\Gamma\tilde{A})- m_y^3+\frac{\tilde{\omega}}{\tilde{\kappa}}\,(-m_y^2+\frac{\tilde{\omega}}{\tilde{\kappa}}\,m_y-\Gamma\tilde{A}+2\,m_y^2)-\frac{\tilde{\omega}^2}{\tilde{\kappa}^2}\,m_y  \\
    &=-\frac{\tilde{\omega}\Gamma}{\sqrt{2}\,\tilde{\kappa}^2}-\frac{\tilde{\omega}}{\tilde{\kappa}}\,\Gamma\tilde{A}\\
    &=-\frac{1}{\sqrt{2}\,\tilde{\kappa}}-\frac{1}{\tilde{\kappa}^2}\,(\Gamma+\gamma-\frac{\tilde{\kappa}}{\sqrt{2}})=-\frac{1}{\tilde{\kappa}^2}\,(\Gamma+\gamma)
\end{align*} 
This can not be satisfied for $\Gamma,\,\gamma\neq0$, which is the case, in which this project is interested in. So the only possible stationary solutions are
\begin{align*}
    m_x&=0\\
    m_y&=\ty\\
    \text{with}\quad0&=-\frac{\tilde{\omega}\Gamma}{\sqrt{2}\,\tilde{\kappa}^2}-\Gamma\tilde{A}\,\ty    - \ty^3+2\,\frac{\tilde{\omega}}{\tilde{\kappa}}\,\ty^2-\frac{\tilde{\omega}^2}{\tilde{\kappa}^2}\,\ty  
\end{align*}
One can multiply the equation with $\tilde{\kappa}^3/\tilde{\omega}^3$ in order to receive
\begin{align*}
    0&=-\frac{\tilde{\kappa}\Gamma}{\sqrt{2}\,\tilde{\omega}^2}-(\frac{\tilde{\kappa}^2}{\tilde{\omega}^2}\,\Gamma\tilde{A}+1)\,\frac{\tilde{\kappa}}{\tilde{\omega}}\,\ty    - \frac{\tilde{\kappa}^3}{\tilde{\omega}^3}\,\ty^3+2\,\frac{\tilde{\kappa}^2}{\tilde{\omega}^2}\,\ty^2 
\end{align*}
By redefining ${y}\vcentcolon=\ty\tilde{\kappa}/\tilde{\omega}$ and $\Gamma A\vcentcolon=\tilde{\kappa}^2/\tilde{\omega}^2\,\Gamma\tilde{A}$, the solutions can be restated as
\begin{align*}
    m_x&=0\\
    m_y&=\tilde{\omega}/\tilde{\kappa}\,{y}\\
    \text{with}\quad0&=-B-(\Gamma A+1)\,{y}    - {y}^3+2\,{y}^2=\vcentcolon F({y})\\
    \text{with}\quad B\vcentcolon&= \frac{\tilde{\kappa}\Gamma}{\sqrt{2}\,\tilde{\omega}^2}
\end{align*}
In a next step it can be examined how many solutions these equations have for ${y}$ have. The polynomial of third degree can have at most two extrema. The derivative of the equation, which determines $y$, with respect to $y$ reads
\begin{align*}
    F'(y)=-\Gamma A-1 -3\,y^2+4\,y
\end{align*}
The roots of this exression are found at
\begin{align*}
    y_{12}=&\frac{1}{3}\,\left(  2\pm \sqrt{4-3\,(\Gamma A+1)}  \right)\\
    =&\frac{1}{3}\,\left(  2\pm \sqrt{1-3\,\Gamma A}  \right)
\end{align*}
Thus $F(y)$ has no extrema, if $\Gamma A\geq1/3$. Through the choice of the parameters $\Gamma A$ the position and existence of the roots can be controlled. By resolving the above equation for $\Gamma A$, and inserting it into $F$, the position of the extrema is determined.
\begin{align*}
    (3\,y_{12}-2)^2=&1-3\,\Gamma A\\
    \Gamma A=& \frac{1}{3}\,\left(  1-(3\,y_{12}-2)^2 \right)\\\\
    \Rightarrow\quad F(y_{12})=&-B-\frac{1}{3}\,\left(  1-(3\,y_{12}-2)^2 +3\right)\,y_{12}-y_{12}^3+2\,y_{12}^2\\
    =&-B-\frac{1}{3}\,\left( - 9\,y_{12}^2+12\,y_{12}\right)\,y_{12}-y_{12}^3+2\,y_{12}^2\\
    =&-B+2\,y_{12}^3-2\,y_{12}^2\\
\end{align*}
$F(y_{12})$ gives the value of the maximum and minimum of $F$. For $B=0$ there remains
\begin{align*}
    F(y_{12})=&2\,y_{12}^3-2\,y_{12}^2\\
    =&2\,y_{12}^2\,(y_{12}-1)\\
    &=\vcentcolon F_\text{max}(y_{12})
\end{align*}
We can tell from here, that the maximum is negative until $y_{12}=1$ which corresponds to $\Gamma A =0$ and the minimum is always negative except when $y_{12}=0$ which corresponds to $\Gamma A=-1$ and $B=0$. \\

\begin{wrapfigure}[19]{l}{0pt}
    \includegraphics{pictures/numb_fixp.pdf}
    \vspace*{-2cm}\caption{The number of fixed points depending on the parameter configuration.}
    \label{fig:numb_fixp}
\end{wrapfigure}
With the rescaling $\Omega=\tw/\sqrt{2\,\Gamma\,(\Gamma+\gamma)}$ and $K=\tk/(\sqrt{2}\,(\Gamma+\gamma))$ we get
\begin{align*}
    B&=\frac{K}{2\,\Omega^2}
\end{align*}
and
\begin{align*}
    y_\text{max}&=\frac{1}{3}\,\left( 2+ \sqrt{1-\frac{3}{2}\,(1-K)\,\frac{1}{\Omega^2}}  \right)
\end{align*}
In this way we can compute the border between the configurations with different numbers of fixed points numerically. This is done by calculating the root of $F(y_{12})$ with respect to $K$ for various values of $\Omega$. As we know from previous considerations if $y_\text{max}<1$, i.e. $\tilde{A}>0$ which corresponds to $K>1$ the maximum of $F$ is negative even for $B=0$, resulting in only one solution. Another thing can be noted at this point. As $F(0)=-B$ and $B\neq0$ for $\kappa\neq0$, one solution $m_y$ has always to be negative for $\kappa\neq0$.\\\\
The next step in the examination is the analysis of the stability of the fixed points.
% \begin{figure}[H]
%     \hspace*{-1cm}
%     \includegraphics{pictures/phaseplot_A_kw.pdf}
%     \caption{The parameter configurations, where there are 1 ore 3 fixed points can be separated.}
%     \label{fig:phases_numb_of_fixp}
% \end{figure}
For this I proceed by calculating the linear expansion of the differential equations.
\begin{align*}
    \dt\left(\begin{array}{c}
         \delta m_x\\
         \delta m_y\\
         \delta m_z
    \end{array}\right)&=\left( \begin{array}{ccc}
        -\Gamma\,A-y^2+\frac{\tilde{\omega}}{\tilde{\kappa}}\,y&  0 & 0\\
        0 & -\Gamma\,A-y^2+\frac{\tilde{\omega}}{\tilde{\kappa}}\,y & \sqrt{2}\,\frac{\Gamma}{\tilde{\kappa}^2}\,(\tilde{\kappa}\,y-\tilde{\omega})\\
        0 &  -\sqrt{2}\,\frac{\Gamma}{\tilde{\kappa}^2}\,(2\tilde{\kappa}\,y-\tilde{\omega}) & -2\,\frac{\Gamma^2}{\tilde{\kappa}^2}
    \end{array} \right)\,\left(\begin{array}{c}
         \delta m_x\\
         \delta m_y\\
         \delta m_z
    \end{array}\right)
\end{align*}
For the analysis of the sign of the eigenvalues one is free to multiply the matrix with a positiv number, i.e $\tilde{\kappa}^2/\tilde{\omega}^2$, what yields the matrix
\begin{align*}
    \mathcal{A}=\left( \begin{array}{ccc}
        -\Gamma A-{y}^2+{y}&  0 & 0\\
        0 & -\Gamma A-{y}^2+{y}& \sqrt{2}\,\Gamma/\tilde{\kappa}\,({y}-\frac{\tilde{\kappa}}{\tilde{\omega}})\\
        0 &  -\sqrt{2}\,\Gamma/\tilde{\kappa}\,(2\,{y}-\frac{\tilde{\kappa}}{\tilde{\omega}}) & -2\,\frac{\Gamma^2}{\tilde{\omega}^2}
    \end{array} \right)
\end{align*}\newpage
The first eigenvalue is already accessible due to the block form of the matrix. In order to determine the sign of the first eigenvalue in general, it is convenient to define an even more general parameter
\begin{wrapfigure}[16]{l}{0pt}
    \includegraphics{pictures/sign_of_ev1.pdf}
    \vspace*{-2cm}\caption{The smaller ($r_{\lambda_1}^{min}$) and larger ($r_{\lambda_1}^{max}$) roots of $\lambda_1$ with marked areas for the value for ${y}=k/w\cdot m_y$. ${y}_{max}$ is the ${y}$-position of $F$.}
    \label{fig:sign_lam1}
\end{wrapfigure}
\begin{align*}
    \chi\vcentcolon&=\frac{K-1}{\Omega^2}\\
    \Rightarrow\quad\Gamma A&=-\half\,\chi\\
    B=\frac{K}{2\,\Omega^2}&=\frac{K-1}{2\,\Omega^2}+\frac{1}{2\,\Omega^2}=\half\,\chi+\frac{1}{2\,\Omega^2}
\end{align*}
The first eigenvalue is in the form of a second order polynomial in $y$ and has therefore in general 2 roots. Between the two roots the eigenvalue is positive marking the fixed point as unstable, where as outside the interval between the roots the eigenvalue is negative, maintaining the possibility of a stable fixed point. When the parameter $\chi$ gets negative \ref{fig:sign_lam1} shows, that the roots of $\lambda_1$ lie both in the region $y>0$. From the previous consideration it is known, that only one fixed point exists for $\chi<0$, i.e $K<1$, and it is negative. Thus for $\chi<0$ the first eigenvalue is always negative.\\\\
Turning to the case $K>1$. Here 3 fixed points are possible. Picturing a possible course of the polynomial of third degree, whose solutions resemble the $y$-values of the fixed points one can make a few statements on the position of those fixed points. The smallest solution, which is always negative, has its maximal value for the shift $B$ being minimal. For a given parameter configuration, it always holds $B<\chi$. So the values of the smallest solution, if $B$ is replaced by $\chi$ are always larger as the true solution. In fact $B\rightarrow\chi$ when $\Omega\rightarrow\infty$ and also $K$ diverges in a way that keeps $\chi$ constant.
\begin{align*}
    \tilde{F}({y})=-\half\,\chi-(1-\half\,\chi)\,{y}    - {y}^3+2\,{y}^2
\end{align*}
% For a given value of $\chi$ the parameter $\Omega$ can be chosen freely, because through an appropriate choice of $K$ the value of $\chi$ can be fixed. So for every possible value of $\chi>0$ the supreme value of $B$ is just $\chi/2$, as $\Omega\rightarrow\infty$.
% The first, smallest fixed point has again because of the positive sign of $B$ always a negative $y$-component. It has it's greatest value for minimal $B$. Negative values of $\chi$ lead to positive roots of $\lambda_1$, which induces that $\lambda_1$ itself is negative. \\\\
It turns out that the roots of $\lambda_1$ are always a root of $\tilde{F}$. Consequently the true $y$-value of the fixed point, with smallest $y$, is always smaller than the root of $\lambda_1$ and therefore this fixed point has a negative first eigenvalue.\\\\
So I'll move on to the next fixed point, the one in the middle. It also takes it's smallest value for minimal $B$, allowing one to look at the function $\tilde{F}$ again. In this constellation the middle root is always at ${y}=1$. The maximal possible $y$-value (actually supremum) for the middle fixed point is at the maximum point of $F$ which is also the minimal possible value of the largest root. The maximum $y$-value of the largest fixed point, is again for minimal $B$ and thus the larger root of $\lambda_1$ is a supremum of this fixed point. \\\\
Summing up the insights of this thoughts, one can state that the first eigenvalue of the smallest $y$-solution is always negative, whereas the eigenvalue of the other possible fixed points is always positive, identifying them as unstable.
% $K$ and $\Omega$ can always be adopted in a way, so that $\chi$ and $1/\Omega^2$ can take every arbitrarily large number.
% \begin{gather*}
%     \frac{K-1}{\Omega^2}=\chi, \quad \frac{1}{\Omega^2}=M\\
%     \Rightarrow K=1+\frac{\chi}{M}\quad\text{and}\quad\Omega = \frac{1}{\sqrt{M}}
% \end{gather*}
% Thus the true $B$ can be made arbitrarily big while $\chi$ stays as it is, allowing in turn, that the fixed point, with the largest $y$-value, gets arbitrarily large.
% So we can already provide a diagramm for the possible sign-values of $\lambda_1$ in \autoref{fig:sign_lam1}.\\\\
\newpage
A question, that comes to mind because of the obscurity of the position of the fixed points, which arises due to the rescaling of $y$, is whether all of the above fixed points are physical, meaning $|m|<=1/2$. This is in fact the case as I show in the Appendix \ref{appendix:mod_of_fixp}. Numerical calculations deliver the same result as shown in the following figure.
\begin{figure}[H]
    \hspace*{-1cm}
    \includegraphics{pictures/fixp_bound_heatmap_ml.png}
    \caption{heatmap of the range of the fixed point for a certain parameter range of $\omega$ and $\kappa$. $\Gamma=1$, $\gamma=0.2$ are fixed. Shown are the middle and larger $m_y$-solutions.}
    \label{fig:fixp_midlarge_bound_hm}
\end{figure}
% \begin{wrapfigure}[22]{l}{0pt}
%     \includegraphics{pictures/fixp_bound_heatmap_s.png}
%     \vspace*{-2cm}\caption{Heatmap for the possible fixed point values for the small $m_y$-solution. (a) depicts $m_y$, whereas (b) depicts $|m|$-values.}
%     
% \end{wrapfigure}
\begin{figure}[H]
    \floatbox[{\capbeside%\captionsetup[capbesidefigure]%{labelsep=newline}%
    \thisfloatsetup{capbesideposition={right,center},capbesidewidth=none}}]{figure}[\FBwidth]
    {\caption{Heatmap for the possible fixed point values for the small $m_y$-solution. (a) depicts $m_y$, whereas (b) depicts $|m|$-values.\\\\
    The fixed points are constraint to the physically allowed space, where $|m|\leq1/2$. In some cases the the boundary is reached. Especially for the middle $m_y$ fixed point the $|m|$ is big for a wide range of parameters. In the other cases $m_y$ goes to zero for $\kappa,\,\omega\rightarrow0$. \\
    As $m_z(m_y\rightarrow0)\rightarrow1/2$, the total collective spin reaches it's boundaries in this limit as well.\\\\ 
    The $m_y=0$ column in the plot for the smallest $m_y$-fixed points stems from the fact that for $\omega=0$ and $A>0$ (i.e $\kappa<\Gamma+\gamma$) $m_y=0$ is the only solution to the stationary problem.}}
    {\includegraphics{pictures/fixp_bound_heatmap_s.png}}
    \label{fig:fixp_small_bound_hm}
    %{\label{fig:num_of_fixp_criterium_BF}}
\end{figure}

Notice that the solutions for $m_y$ in these plots are determined from a more original equation in order to avoid unnecessary pols.
\begin{align*}
    0=&-\half\,\Gamma\omega-(\half\,\Gamma\,(\Gamma+\gamma-\kappa)+\omega^2)\,m_y\\&-\kappa^2\,m_y^3+2\,\kappa\omega\,m_y^2
\end{align*}
The remaining analysis will be performed numerically. Thus there is less need for rescaling quantities, instead it is often more convenient to stick to the original form. In this form the linearization of the equations of motion result in the following matrix.
\begin{align*}
    \Gamma\,\mathcal{A}=&\left( \begin{array}{c c c}
        -\half\,\Gamma\,(\Gamma+\gamma-\kappa)-\kappa^2\,m_y^2
        +\omega\kappa\,m_y&0&0\\
        0&-\half\,\Gamma\,(\Gamma+\gamma-\kappa)-\kappa^2\,m_y^2
        +\omega\kappa\,m_y&\Gamma\,(\kappa\,m_y-\omega)\\
        0&-\Gamma\,(2\kappa\,m_y-\omega)&-\Gamma^2
    \end{array}  \right)
\end{align*}
\begin{wrapfigure}[24]{l}{0pt}
    \includegraphics{pictures/lam2_anal_s.png}
    \vspace*{-2cm}\caption{Heatmap for the possible fixed point values for the small $m_y$-solution. (a) depicts $m_y$, whereas (b) depicts $|m|$-values.}
    \label{fig:sign_lam23_s}
\end{wrapfigure}
The other two eigenvalues of the linearization matrix are the t-roots of the expression
\begin{align*}
    0=&t^2+\left( \Gamma^2 +\half\,\Gamma\,(\Gamma+\gamma-\kappa)+\kappa^2\,m_y^2
    -\omega\kappa\,m_y\right)\,t\\
    &+\Gamma^2\,(\half\,\Gamma\,(\Gamma+\gamma-\kappa)+\kappa^2\,m_y^2-\omega\kappa\,m_y)\\
    &+\Gamma^2\,(\kappa\,m_y-\omega)\,(2\,\kappa\,m_y-\omega)
    &=\vcentcolon t^2+p\,t+q
\end{align*}
or
\begin{align*}
    0=&t^2+\left( 2\,\frac{\Gamma^2}{\tilde{\omega}^2}+\Gamma\tilde{A}+y^2-y \right)\,t\\
    &+2\,\frac{\Gamma^2}{\tilde{\omega}^2}\,\left( \Gamma\tilde{A}+y^2-y \right)\\
    &+2\,\frac{\Gamma^2}{\tilde{\kappa}^2}\,(2\,y-\frac{\tilde{\kappa}}{\tilde{\omega}})\,(y-\frac{\tilde{\kappa}}{\tilde{\omega}})\\
    &=\vcentcolon t^2+p\,t+q
\end{align*}
Starting with the analysis of $p$ and going back one last time in the rescaled form with the above defined new parameters, one finds
\begin{align*}
    p=&\frac{\Gamma}{\Omega^2\,(\Gamma+\gamma)}-\half\,\chi+3\,y^2-4\,y+1
\end{align*}
the roots of this expression, which are maximally apart from each other, are found, when $\Omega\rightarrow\infty$ in the same spirit as above. This yields as roots of $p$
\begin{align*}
    r_p=&\frac{1}{3}\,\left( 2\pm\sqrt{1+\frac{3}{2}\,\chi-3\,\frac{\Gamma}{\Omega^2\,(\Gamma+\gamma)}} \right)\\
    r_p^\text{max}=&\frac{1}{3}\,\left( 2+\sqrt{1+\frac{3}{2}\,\chi} \right)\\
    r_p^\text{min}=&\frac{1}{3}\,\left( 2-\sqrt{1+\frac{3}{2}\,\chi} \right)
\end{align*}\\\\
We can recognize, that $r_p^\text{max}=\tilde{y}_\text{max}$ and $r_p^\text{min}=\tilde{y}_\text{min}$. Thus we can infer that for the bracketing fixed points $p$ is always positive, whereas for the middle fixed point we can't make this statement.\\\\
This information is retrieved by numerical computation of the roots. The eigenvalues can be calculated via the expression
\begin{align*}
    \lambda_{23}=&\half\,(-p\pm\sqrt{p^2-4q})\quad.
\end{align*}
If the eigenvalues become a complex number, it allows, that the solution of the full Dynamics have oscillating character near those fixed point. Complex eigenvalues are reached when the inside of the square root becomes negative. We define
\begin{align*}
    \tilde{q}\vcentcolon=p^2-4q\quad.
\end{align*}
To describe the properties of the fixed points we investigate three values $p$, $q$ and $\tilde{q}$. As already mentioned a negative $\tilde{q}$ signals complex eigenvalues. In this case $-p$ is the real part of the eigenvalue. On the other hand when $q$ gets negative the eigenvalues $\lambda_2$ and $\lambda_3$ have different signs, whereas when $q\geq0$ the eigenvalues $\lambda_{23}$ have always the oposite sign of $p$



\begin{figure}[H]
    \centering
    \includegraphics{pictures/lam2_anal_ml.png}
    \caption{The yellow line signals in each plot the border between the area with one vs. more than one fix points. Above the line there is only one fixed point. The figures a-c present the sign of the interior of the square root for the eigenvalues $\lambda_{23}$ for different parameter configurations-in the lower segment of the plots for the fixed point specified int the subcaption, in the upper segment for the only fixed point, which is why the plots look the same in this part. Figures d-f show in the same manner the sign of $p$. In the entire setup is $\Gamma=1$ and $\gamma=0.2$ fixed.}
\end{figure}
\newpage
After completed the discussion for $\delta=0$ we can choose non-zero $\delta$. In this discussion we rely on numerical calculation for our examinations, because of the complexity of the equation. The strategy for analysing this parameter regime stays the same. I determine the roots of the equations
\begin{align*}
    -\Gamma\delta\,m_y-\Gamma\,(\gamma+\Gamma-\kappa)\,m_x-2\,\kappa^2\,m_x\,( m_x^2+ m_y^2)+2\,\kappa\omega\,m_xm_y  &=0\\\\
    \Gamma\delta\,m_x-\omega\Gamma-\Gamma\,(\gamma+\Gamma-\kappa)\,m_y-2\,\omega^2\,m_y&\\
    -2\,\kappa^2\,m_y\,( m_x^2+ m_y^2)+2\,\kappa\omega\,(m_x^2+2\,m_y^2)  &=0
\end{align*}
If we multiply the first equation by $m_y$ and the second by $m_x$ and subtract the two equations from each other we get
\begin{align*}
    -\Gamma\delta\,(m_x^2+m_y^2)+\Gamma\omega\,m_x+2\,\omega^2\,m_xm_y-2\,\kappa\omega\,m_x\,(m_x^2+m_y^2)
\end{align*}
By solving this equation for $m_y$ and plugging the result into the second equation, one has two find a root of a one variable function. Then I divided the x-axis into a grid and searched bewtween the gridpoints for roots in order find all of the fixed points.
\begin{align*}
    m_{y,12}(m_x)=\frac{1}{\Gamma\delta+2\,\kappa\omega\,m_x}\,\left( \omega^2\,m_x\pm \sqrt{\omega^4\,m_x^2-(\Gamma\delta+2\,\kappa\omega\,m_x)\,[(\Gamma\delta+2\,\kappa\omega\,m_x)\,m_x^2-\Gamma\omega\,m_x]} \right)
\end{align*}
\begin{figure}[H]
    % \vspace*{-1cm}
    \hspace*{-1.2cm}
    \includegraphics{pictures/numb_of_fixp.png}
    \caption{The blue filled region has 3 fixed points, whereas the rest of the shown parameter space has 1 fixed point. Shown are different views.}
\end{figure}\newpage
After obtaining the fixed points I linearized again the differential equation and determined numerically the eigenvalues of the evolving matrix. From the signs of the eigenvalues there can be retrieved information about the stability of the fixed points and therefore the state of the system for long times. In the regions where there is no stable fixed point, one expects in the absense of chaos the existance of a limit cicle, which is also confirmed by integrating the equation of motion for a random sample of paramter specifications.

\begin{appendices}

\section{Derivation of the equations of motion}
\label{appendix:eqm_derv}

For the calculations with Pauli matrices revise a few of their properties.
\begin{align*}
    [\sigma^\alpha,\sigma^\beta]&=2i\,\varepsilon_{\alpha\beta\lambda}\,\sigma^\lambda\\
    \Rightarrow\quad[J_\alpha,J_\beta]&=i\,\varepsilon_{\alpha\beta\lambda}\,J_\lambda\\
    [\sigma^z,\sigma^\pm]&=\pm2\,\sigma^\pm\\
    \Rightarrow\quad[J_z,J_\pm]&=\pm J_\pm\\
    \sigma^x\sigma^z&=\left( \begin{array}{cc}
         0 & -1  \\
         1& 0
    \end{array}\right)=-i\,\sigma^y\\
    \sigma^y\sigma^z&=\left( \begin{array}{cc}
         0 & i  \\
         i& 0
    \end{array}\right)=i\,\sigma_x\\
    \sigma^x\sigma^y&=\left( \begin{array}{cc}
         i & 0  \\
         0 & -i
    \end{array}\right)=i\,\sigma_z\\
    \sigma^\pm\sigma^z&=-i\,\sigma^y\pm i^2\,\sigma^x=\mp\sigma^\pm\\
    \sigma^-\sigma^+&=(\sigma^x)^2+(\sigma^y)^2+i\,\sigma^x\sigma^y-i\,\sigma^y\sigma^x\\
    &=2-2\,\sigma^z\\
    \sigma^+\sigma^-&=(\sigma^x)^2+(\sigma^y)^2-i\,\sigma^x\sigma^y+i\,\sigma^y\sigma^x\\
    &=2+2\,\sigma^z\\
    [\sigma^+,\sigma^-]_-&=4\,\sigma^z
\end{align*}
From this the different contributions to the Lindblad follow.
\begin{align*}
    \mathcal{L}_S(J_z)&=-i\,\omega\,\braket{[J_z,J_x]}\\
    &=\omega\,\braket{J_y}\\
    \mathcal{L}_S(J_\pm)&=-i\,\frac{\delta}{2}\,\braket{[J_\pm,J_z]}-i\,\omega\,\braket{[J_\pm,J_x]}\\
    &=-i\,\left(\mp\frac{\delta}{2}\, \braket{J_\pm} \pm \omega\, \braket{J_z}\right)
\end{align*}

\begin{align*}
    \mathcal{L}_a(J_z)&=\frac{\kappa}{N}\,\Trs{\left(J_+ J_z J_- \rho-\half\,[J_z,J_+ J_-]_+\,\rho\right)}\\
    &=\frac{\kappa}{N}\,\Trs{\left(J_z J_+ J_- - J_+ J_- -\half\,(J_z J_+ J_- + J_+ J_z J_- + J_+ J_-)\right)\,\rho}\\
    &=\frac{\kappa}{N}\,\Trs{\left(J_z J_+ J_- - J_+ J_- -\half\,(2\,J_z J_+ J_- + J_+ J_- - J_+ J_-)\right)\,\rho}\\
    &=-\frac{\kappa}{N}\,\Trs{J_+ J_- \rho}\\\\
    \mathcal{L}_a(J_+)&=\frac{\kappa}{N}\,\Trs{\left(J_+ J_+ J_- \rho-\half\,[J_+,J_+ J_-]_+\,\rho\right)}\\
    &=\frac{\kappa}{N}\,\Trs{\left(J_+ J_- J_+ + 2\,J_+ J_z -\half\,(J_+ J_- J_+ + J_+ J_+ J_-)\right)\,\rho}\\ &=\frac{\kappa}{N}\,\Trs{\left(J_+ J_- J_+ + 2\,J_+ J_z -\half\,(2\,J_+ J_- J_+ + 2 J_+ J_z)\right)\,\rho}\\\\
    &=\frac{\kappa}{N}\,\Trs{J_+ J_z \rho}\\\\
    \mathcal{L}_a(J_-)&=\frac{\kappa}{N}\,\Trs{\left(J_+ J_- J_- \rho-\half\,[J_-,J_+ J_-]_+\,\rho\right)}\\
    &=\frac{\kappa}{N}\,\Trs{\left(J_- J_+ J_- + 2\,J_z J_- -\half\,(J_- J_+ J_- + J_+ J_- J_-)\right)\,\rho}\\ 
    &=\frac{\kappa}{N}\,\Trs{J_z J_- \rho}
\end{align*}
\begin{align*}
    \mathcal{L}_b(J_z)&=0\\
    \mathcal{L}_b(J_\pm)&=\gamma\,\Trs{\left(J_z J_\pm J_z \rho-\half\,[J_\pm,J_z^2]_+\,\rho\right)}\\
    &=\gamma\,\Trs{\left(J_\pm J_z^2\rho \pm J_\pm J_z \rho-\half\,(J_\pm J_z^2 + J_z J_\pm J_z \pm J_z J_\pm)\,\rho\right)}\\
    &=\gamma\,\Trs{\left(J_\pm J_z^2\rho \pm J_\pm J_z \rho-\half\,(2\,J_\pm J_z^2  \pm J_z J_\pm \pm J_\pm J_z)\,\rho\right)}\\
    &=\pm\half\, \gamma\,\Trs{[J_\pm,J_z]\,\rho}\\
    &=-\half\,\gamma\,\Trs{J_\pm\rho}
\end{align*}
\begin{align*}
    \mathcal{L}_c(J_z)&=\frac{\Gamma}{8}\,\sum_{j,k=1}^N \Trs{\left( \sigma_j^- \sigma_k^z \sigma_j^+ -\half\,[\sigma_k^z,\sigma_j^-\sigma_j^+]_+   \right)\,\rho}\\
    &=\frac{\Gamma}{8}\,\sum_{j,k=1}^N \Trs{\left( \sigma_j^- \sigma_k^z \sigma_j^+ -\half\,[\sigma_k^z\sigma_j^-\sigma_j^++\sigma_j^-\sigma_j^+\sigma_k^z]   \right)\,\rho}\\
    &=\frac{\Gamma}{8}\,\sum_{j,k=1}^N \Trs{\left( \sigma_j^- \sigma_k^z \sigma_j^+ -\half\,[2\,\sigma_j^-\sigma_k^z\sigma_j^+  -2\, \sigma_j^-\sigma_j^+\delta_{jk}-2\, \sigma_j^-\sigma_j^+\delta_{jk}]   \right)\,\rho}\\
    &=\frac{\Gamma}{8}\,\sum_{k=1}^N \Trs{2\, \sigma_k^-\sigma_k^+  \,\rho}\\
    &=\frac{\Gamma}{4}\,\sum_{k=1}^N \Trs{ (2-2\,\sigma_k^z)  \,\rho}\\
    &=\half\,N\,\Gamma-\Gamma\,\braket{J_z}
\end{align*}
\begin{align*}
    \mathcal{L}_c(J_-)&=\frac{\Gamma}{4}\,\sum_{j,k=1}^N \Trs{\left( \sigma_j^- \sigma_k^- \sigma_j^+ -\half\,[\sigma_k^-,\sigma_j^-\sigma_j^+]_+   \right)\,\rho}\\
    &=\frac{\Gamma}{4}\,\sum_{k=1}^N \Trs{\left( \sigma_k^- \sigma_k^- \sigma_k^+ -\half\,[\sigma_k^-\sigma_k^-\sigma_k^++\sigma_k^-\sigma_k^+\sigma_k^-]   \right)\,\rho}\\
    &=-\frac{\Gamma}{2}\,\sum_{k=1}^N\Trs{\sigma_k^-\sigma_k^z\rho}\\
    &=-\frac{\Gamma}{2}\,\sum_{k=1}^N\Trs{\sigma_k^-\rho}\\\\
    \mathcal{L}_c(J_+)&=\frac{\Gamma}{4}\,\sum_{j,k=1}^N \Trs{\left( \sigma_j^- \sigma_k^+ \sigma_j^+ -\half\,[\sigma_k^+,\sigma_j^-\sigma_j^+]_+   \right)\,\rho}\\
    &=-\frac{\Gamma}{2}\,\sum_{k=1}^N\Trs{\sigma_k^z\sigma_k^+\rho}\\
    &=-\frac{\Gamma}{2}\,\sum_{k=1}^N\Trs{\sigma_k^+\rho}
\end{align*}

\section{The J-square calculus}
\label{appendix:msq_calc}
If there is a system of $N$ coupled ODE, this could be seen as the quest for a function $\vec{F}:\,\mathbb{R}\rightarrow\mathbb{R}^N$. Derived from that, there can also be defined the function squared $F^2:\,\mathbb{R}\rightarrow\mathbb{R}:\,t\mapsto\sum_{i=1}^NF_i^2(t)$. If one is now able to write the derivative of squared function in the following way
\begin{align*}
    \dt F^2(t)&=\vec{F}^t(t)\,A\,\vec{F}(t)
\end{align*}
with a Matrix $A$, which is positive ore negative definite, then there do not exist stationary solutions to the system of ODE, including limit cycles. \\
\textit{Proof}: The functions $\vec{F}$, can be mapped via N-dimensional spherical coordinates into a different set of functions $f(t)$, $\varphi_j(t)$ $j\in\{1,\dots,N-1\}$, with $f$ the modulus of $F$, and $\varphi_j$ the angles. This is true for all solutions $\vec{F}\neq\vec{0}$. In the following we neglect this special case, as this is in the most cases anyway a stationary point. With the notation.
\begin{align*}
    \vec{F}(t)&=f(t)\,\hat{r}(t)
\end{align*}
we can rewrite the differential equation for the squared function to
\begin{align*}
    \dt F^2(t)=\dt f^2(t)=2\,f(t)\,\dt f(t)&=f^2(t)\,\hat{r}^t\,A\,\hat{r}\\
    \Rightarrow\quad\dt f(t) = f(t)\,\hat{r}^t\,A\,\hat{r}
\end{align*}
Assuming that A is positive or negative definite we can use the property that $|\hat{r}(t)|=1\,\forall t$, in order to follow, that there exists an $\varepsilon>0$ so that
\begin{align*}
    \lambda(t)\vcentcolon=\left| \hat{r}^t(t)\,A\,\hat{r}(t) \right| > \varepsilon
\end{align*}
For all $t$ and all functions $\{\varphi_j(t)\}$. \textit{Proof}: if this doesn't hold, there will have to be a sequence $(t_n)$ with $\lambda(t_n)\rightarrow0$. As the surface of an N-dimensional sphere is a compact set, this sequence would have to converge on the survace of the sphere, which would infer $\exists \hat{r}\neq0$ with 
\begin{align*}
    \hat{r}^t(t)\,A\,\hat{r}(t)=0
\end{align*}
in contradiction to the definiteness of $A$. So depending on, whether $A$ is positive or negative definite, we can write
\begin{align*}
    \dt f(t) >& \varepsilon\,f(t)\\
    \Rightarrow\quad f(t) >& c\,e^{\varepsilon\,t}\\
    \text{or}\quad \dt f(t) <& -\varepsilon\,f(t)\\
    \Rightarrow\quad f(t) <& c\,e^{-\varepsilon\,t}\\
\end{align*}
for suitable choice of $c$ (can be shown with mean value theorem). This implies, that there can't be a stationary solution of the set of ODE. The last statement holds, because for $t\rightarrow\infty$ either $f$ goes to zero, which would imply that all $F_i$ go to zero or $f$ grows boundlessly, which implies that, independent of the starting conditions, there exists a $F_i$, that grows boundlessly, which forbids the properties of a stationary state including limit cycles.

\section{Modulus of the fixed points in the case $\delta=0$}
\label{appendix:mod_of_fixp}
\begin{align*}
    m_z&=\half-\frac{1}{\Gamma}\,\left( \kappa\,(m_x^2+m_y^2)-\omega m_y  \right)\\
    &=\half-\frac{\kappa}{\Gamma}\,\left( m_y^2-\frac{\omega}{\kappa}\, m_y  \right)\\
    \Rightarrow\quad |m|^2&=\left( \half-\frac{\kappa}{\Gamma}\,\left( m_y^2-\frac{\omega}{\kappa}\, m_y  \right) \right)^2+m_y^2\\
    &=\left( \half-\frac{\omega^2}{\Gamma\kappa}\,\left( {y}^2- {y} \right) \right)^2+\frac{\omega^2}{\kappa^2}\,{y}^2\\
    &=\left( \half-\frac{\Omega^2}{K}\,\left( {y}^2- {y} \right) \right)^2+\frac{\Gamma}{\Gamma+\gamma}\,\frac{\Omega^2}{K^2}\,{y}^2\\
    \leq&\left( \half-\frac{\Omega^2}{K}\,\left( {y}^2- {y} \right) \right)^2+\frac{\Omega^2}{K^2}\,{y}^2\\
    &=\vcentcolon m^*
\end{align*}

Now I set $\Omega^2/K$ through the determining equation for $m_y$.
\begin{align*}
    0&=-\frac{K}{2\Omega^2}-\left( \frac{1-K}{2\Omega^2} +1\right)\,{y}-{y}^2+2\,{y}^2\\
    \Rightarrow\quad \frac{K}{2\Omega^2}\,({y}-1)&=(1+\frac{1}{2\Omega^2})\,{y}+{y}^3-2\,{y}^2\\
    \frac{\Omega^2}{K}&=\frac{{y}-1}{2\,\left(  (1+\frac{1}{2\Omega^2})\,{y}+{y}^3-2\,{y}^2\right)}
\end{align*}
\begin{align*}
    \frac{\text{d}}{\text{d}{y}}\,\frac{\Omega^2}{K}&=\frac{1}{2\,\left(  (1+\frac{1}{2\Omega^2})\,{y}+{y}^3-2\,{y}^2\right)}-\frac{2\,({y}-1)\,(1+\frac{1}{2\Omega^2}+3\,{y}^2-4\,{y})}{4\,\left(  (1+\frac{1}{2\Omega^2})\,{y}+{y}^3-2\,{y}^2\right)^2}\\\\
    &=\frac{1+\frac{1}{2\Omega^2}-4\,{y}+5\,{y}^2-2\,{y}^3}{2\,\left(  (1+\frac{1}{2\Omega^2})\,{y}+{y}^3-2\,{y}^2\right)^2}
\end{align*}

This equation looks at first strange, because $k/2\Omega^2$ could be negative for $0<{y}<1$. But as we've already seen in \autoref{fig:sign_lam1}, this space is free of fixed points. \\\\
Taking the derivative of $m^*$
\begin{align*}
    \frac{\partial m^*}{\partial{y}}&=-2\,\left(\frac{\Omega^2}{K}\,(2{y}-1)+\left(\frac{\text{d}}{\text{d}{y}}\,\frac{\Omega^2}{K}\right)\,({y}^2-{y})\right)\,\left( \half-\frac{\Omega^2}{K}\,\left( {y}^2- {y} \right) \right)+2\,\frac{\Omega^2}{K^2}\,{y}+2\,\left(\frac{\text{d}}{\text{d}{y}}\,\frac{\Omega^2}{K}\right)\,\frac{\Omega^2}{K}\,\frac{1}{\Omega^2}\,{y}^2\\\\
    &=4\,(\frac{\Omega^2}{K})^2\,{y}^3-\left(2\,(\frac{\Omega^2}{K})^2+4\,(\frac{\Omega^2}{K})^2\right)\,{y}^2+\left( 2\,(\frac{\Omega^2}{K})^2+2\,\frac{\Omega^2}{K^2}-2\,\frac{\Omega^2}{K} \right)\,{y}+\frac{\Omega^2}{K}\\\\
    &+2\,\dkw\,\kw\,{y}^4-4\,\dkw\,\kw\,{y}^3\\\\
    &-\left(\dkw-2\,\dkw\,\kw-2\,\dkw\,\kw\,\frac{1}{\Omega^2}\right)\,{y}^2+\dkw\,{y}\\\\
    &=\frac{\Omega^2}{K}\,\left[  4\,\frac{\Omega^2}{K}\,{y}^3-6\,\frac{\Omega^2}{K}\,{y}^2+\left( 2\,\frac{\Omega^2}{K}+2\,\frac{\Omega^2}{K}\,\frac{1}{\Omega^2}-2 \right)\,{y}+1 \right]\\\\
    &+\dkw\,\left[2\,\kw\,{y}^4-4\,\kw\,{y}^3-\left(1-2\,\kw-2\,\kw\,\frac{1}{\Omega^2}\right)\,{y}^2+{y}\right]\\\\
    &=\vcentcolon \frac{\partial \alpha}{\partial{y}}+\dkw\,\frac{\partial \beta}{\partial{y}}
\end{align*}

So plugging the found result into the equation for the derivative of $m^*$ yields
\begin{align*}
    \frac{2\,\left[(1+\frac{1}{2\Omega^2})\,{y}+{y}^3-2\,{y}^2\right]^2}{{y}-1}\,\frac{\partial \alpha}{\partial{y}}&=2\,({y}-1)\,{y}^3-3\,({y}-1)\,{y}^2\\
    &+\left( {y}-1+({y}-1)\,\frac{1}{\Omega^2}-2\,\left((1+\frac{1}{2\Omega^2})\,{y}+{y}^3-2\,{y}^2\right) \right)\,{y}\\
    &+(1+\frac{1}{2\Omega^2})\,{y}+{y}^3-2\,{y}^2 \\\\
    &=3\,{y}^2+\left( -(1+\frac{1}{\Omega^2})-{y}\right)\,{y}+(1+\frac{1}{2\Omega^2})\,{y}-2\,{y}^2\\
    &=-\frac{1}{2\Omega^2}\,{y}\\\\
    \Rightarrow\quad\frac{\partial \alpha}{\partial{y}}&=-\frac{{y}-1}{2\,\left[(1+\frac{1}{2\Omega^2})\,{y}+{y}^3-2\,{y}^2\right]^2}\,\frac{1}{2\Omega^2}\,{y}
\end{align*}
Doing the same for the $\beta$-expression
\begin{align*}
    \left[(1+\frac{1}{2\Omega^2})\,{y}+{y}^3-2\,{y}^2\right]\,\frac{\partial \beta}{\partial{y}}&=({y}-1)\,{y}^4-2\,({y}-1)\,{y}^3-\left( (1+\frac{1}{2\Omega^2})\,{y}+{y}^3-2\,{y}^2-{y}+1-({y}-1)\,\frac{1}{\Omega^2} \right)\,{y}^2\\
    &+((1+\frac{1}{2\Omega^2})\,{y}+{y}^3-2\,{y}^2)\,{y}\\
    &=-\left( 1+\frac{1}{\Omega^2}-\frac{1}{2\Omega^2}\,{y}  \right)\,{y}^2+(1+\frac{1}{2\Omega^2})\,{y}^2\\
    &=\frac{1}{2\Omega^2}\,({y}^3-{y}^2)=\frac{1}{2\Omega^2}\,{y}^2\,({y}-1)
\end{align*}
Collecting the terms together, it is found
\begin{align*}
    \frac{\partial \alpha}{\partial{y}}+\dkw\,\frac{\partial \beta}{\partial{y}}&=-\frac{\frac{1}{2\Omega^2}\,({y}-1)}{2\,\left[(1+\frac{1}{2\Omega^2})\,{y}+{y}^3-2\,{y}^2\right]^3}\,\left( {y}\,((1+\frac{1}{2\Omega^2})\,{y}+{y}^3-2\,{y}^2) -{y}^2\,(1+\frac{1}{2\Omega^2}-4\,{y}+5\,{y}^2-2\,{y}^3)\right)\\
    &=-\frac{\frac{1}{2\Omega^2}\,({y}-1)}{2\,\left[(1+\frac{1}{2\Omega^2})\,{y}+{y}^3-2\,{y}^2\right]^3}\,\left( 2\,{y}^5-4\,{y}^4+2\,{y}^3 \right)\\
    &=-\frac{\frac{1}{2\Omega^2}\,({y}-1)}{2\,\left[(1+\frac{1}{2\Omega^2})+{y}^2-2\,{y}\right]^3}\,\left( 2\,{y}^2-4\,{y}+2 \right)
\end{align*}
We can examine the roots of the two quadratic polynomials in numerator and denominater
\begin{align*}
    2\,{y}^2-4\,{y}+2\rightarrow\text{roots}&=\frac{1}{4}\,(4\pm\sqrt{16-16})\\
    &=1\\
    \Rightarrow\quad 2\,{y}^2-4\,{y}+2&=2\,({y}-1)^2\\
    (1+\frac{1}{2\Omega^2})+{y}^2-2\,{y}\rightarrow\text{roots}&=\half\,(2\pm\sqrt{4-4\,(1+\,\frac{1}{2\Omega^2})})\\
    &=2\pm\sqrt{-\frac{1}{2\Omega^2}}
\end{align*}
So the denominator is always larger then zero
\begin{align*}
    \frac{\partial m^*}{\partial{y}}&=-\frac{\frac{1}{2\Omega^2}\,({y}-1)^3}{\left[(1+\frac{1}{2\Omega^2})+{y}^2-2\,{y}\right]^3}
\end{align*}
We see here that $\partial_{{y}}\,m^*>0$ for ${y}<1$ and $\partial_{{y}}\,m^*<0$ for ${y}>1$. So for the two areas in which ${y}$ can fall, for the one smaller than zero $m^*$ takes it's maximum for ${y}\rightarrow0$ and for the area ${y}>1$ $m^*$ takes it's maximum for ${y}\rightarrow1$. Looking again of the form of $|m|^2$ 
\begin{align*}
    |m|^2\leq m^*\left( \half-\frac{\Omega^2}{K}\,\left( {y}^2- {y} \right) \right)^2+\frac{\Omega^2}{K^2}\,{y}^2
\end{align*}
we can say $m^*({y}\rightarrow0)=1/4$. The limit of ${y}\rightarrow1$ can only be achieved in the limit $\Omega\rightarrow\infty$, while% i.e. $\Gamma/\omega^2\rightarrow0$, while
\begin{align*}
    \text{constant}=\frac{K}{\Omega^2}=\frac{\kappa\Gamma}{\omega^2}
\end{align*}
This means in particular, that also $K\rightarrow\infty$ and therefore $K^2/\Omega^2\rightarrow\infty$. But this fixes $|m|$ to $1/2$. 
\begin{figure}[H]
    \hspace*{-1cm}
    \includegraphics{pictures/fixp_boundaries.pdf}
    \caption{This graphic showes the range of $m_y$, $m_z$, $|m|$ for a parameter grid of $1.2\leq\kappa\leq24$ and $3\cdot10^{-7}\leq\omega\leq7$, $\Gamma=1$ $\gamma=0.2$. In the left column (a-d) the fixed points are shown in the $y-z$-plane, whereas in the right column (e-h) the modulus of the total angular momentum is depectid in dependence of $m_y$ for the different parameter values. The first row shows the fixed point in the parameter regime, where there is only one fixed point. From the 2. row downwards the fixed points in the area of three stationary solutions are show in the order smallest, middle and largest $m_y$-value.}
\end{figure}

\end{appendices}

\printbibliography

\end{document}
% Both can be implemented through stationary electric and magnetic fields, which are described in a semi-classical way. The possibility to treat the field classically arises, when the fields are very off resonant to the atomic transition considered. This is certainly the case if the field is assumed to be static. In this case the Zeeman and Stark effect come into play, which are described through a classical field.